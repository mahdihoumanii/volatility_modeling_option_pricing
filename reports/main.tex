\documentclass[11pt,a4paper]{article}

\usepackage{amsmath,amssymb}
\usepackage{graphicx}
\usepackage{booktabs}
\usepackage{geometry}
\usepackage{hyperref}

\geometry{margin=1in}

\title{\textbf{Volatility Modelling and Option Pricing}\\
\vspace{0.2cm}
\large A Quantitative Study Using Rolling, EWMA, and GARCH Models}
\author{Mohamad ElMahdi Houmani}
\date{\today}

\begin{document}
\maketitle

\section{Data}
We study daily adjusted close prices of the SPY ETF over the period 2015--2024.
Prices are converted to log returns,
\[
r_t = \log\left(\frac{P_t}{P_{t-1}}\right),
\]
which form the basic input for all volatility models.
Returns are treated as approximately mean-zero and are expressed in daily units unless stated otherwise.
Annualization is performed using a factor of $\sqrt{252}$.

\section{Volatility Models}
We estimate time-varying volatility using three standard approaches:
\begin{itemize}
    \item \textbf{Rolling (Historical) Volatility:} Standard deviation of returns over a fixed window (20 trading days).
    \item \textbf{EWMA Volatility:} Exponentially weighted moving average variance,
    \[
    \sigma_t^2 = \lambda \sigma_{t-1}^2 + (1-\lambda) r_{t-1}^2,
    \quad \lambda = 0.94.
    \]
    \item \textbf{GARCH(1,1):} Conditional variance model,
    \[
    \sigma_t^2 = \omega + \alpha r_{t-1}^2 + \beta \sigma_{t-1}^2,
    \]
    with parameters estimated via maximum likelihood.
\end{itemize}
All models produce one-step-ahead volatility forecasts.

\section{Evaluation Methodology}
Volatility forecasts are evaluated using a walk-forward (out-of-sample) framework.
At each time $t$, models are fitted using data available up to $t$ and used to forecast volatility for $t+1$.

To reduce noise in the target, realized volatility is computed using windowed realized variance,
\[
\text{RV}^{(w)}_t = \sqrt{\frac{252}{w} \sum_{i=0}^{w-1} r_{t-i}^2},
\]
with windows $w=5$ (weekly) and $w=21$ (monthly).

Forecast accuracy is assessed using standard metrics:
\begin{itemize}
    \item Mean Squared Error (MSE) on variance,
    \item Mean Absolute Error (MAE) on volatility,
    \item QLIKE loss, commonly used in volatility forecasting.
\end{itemize}

\section{Results}

\subsection*{Forecast Accuracy}
Table~\ref{tab:metrics} summarizes the out-of-sample performance for each model.
Smoothed realized volatility leads to more stable and interpretable comparisons.

\begin{table}[h]
\centering
\caption{Out-of-sample volatility forecast evaluation using windowed realized volatility}
\label{tab:metrics}
\begin{tabular}{llccc}
\toprule
Model & Window & MSE (variance) & MAE (vol) & QLIKE \\
\midrule
Historical & 5  & 0.009359 & 0.052850 & -2.713700 \\
Historical & 21 & 0.000178 & 0.007565 & -2.767174 \\
EWMA       & 5  & 0.008596 & 0.056341 & -2.683947 \\
EWMA       & 21 & 0.000927 & 0.021133 & -2.732824 \\
GARCH(1,1) & 5  & 0.001597 & 0.032931 & -2.838724 \\
GARCH(1,1) & 21 & 0.004505 & 0.033791 & -2.679181 \\
\bottomrule
\end{tabular}
\end{table}

\subsection*{Forecasted vs Realized Volatility}
Figure~\ref{fig:forecast} compares forecasted volatility to windowed realized volatility.
All models capture volatility clustering, with GARCH reacting most strongly to shocks and exhibiting clear mean reversion.

\begin{figure}[h]
\centering
\includegraphics[width=0.95\textwidth]{figures/forecast_vs_realized.png}
\caption{Forecasted volatility vs windowed realized volatility (annualized)}
\label{fig:forecast}
\end{figure}

\section{Option Pricing}
Estimated volatility is used as input for European call option pricing.

\begin{itemize}
    \item \textbf{Black--Scholes:} Closed-form pricing under constant volatility.
    \item \textbf{Monte Carlo:} Simulation of geometric Brownian motion under the risk-neutral measure,
    \[
    dS_t = r S_t dt + \sigma S_t dW_t.
    \]
\end{itemize}

Monte Carlo pricing includes 95\% confidence intervals,
demonstrating convergence to the Black--Scholes price at the expected rate $\mathcal{O}(1/\sqrt{N})$.

\begin{figure}[h]
\centering
\includegraphics[width=0.95\textwidth]{figures/mc_ci_convergence.png}
\caption{Monte Carlo option pricing with 95\% confidence intervals}
\label{fig:mc}
\end{figure}

\section{Limitations}
Several limitations should be noted:
\begin{itemize}
    \item Daily data limits the precision of realized volatility estimation.
    \item Windowed realized variance remains a proxy for latent volatility.
    \item Black--Scholes and Monte Carlo pricing assume lognormal dynamics and constant volatility.
    \item Transaction costs and market microstructure effects are not modeled.
\end{itemize}

Despite these limitations, the project demonstrates a complete and coherent volatility modelling and option pricing pipeline.

\end{document}